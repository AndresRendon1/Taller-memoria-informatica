\documentclass{article}
\usepackage[utf8]{inputenc}
\usepackage[spanish]{babel}
\usepackage{listings}
\usepackage{graphicx}
\graphicspath{ {images/} }
\usepackage{cite}

\begin{document}

\begin{titlepage}
    \begin{center}
        \vspace*{1cm}
            
        \Huge
        \textbf{Proyecto De Investigación}
            
        \vspace{0.5cm}
        \LARGE
        Taller Memoria
            
        \vspace{1.5cm}
            
        \textbf{Andrés Felipe Rendón Villada}
            
        \vfill
            
        \vspace{0.8cm}
            
        \Large
        Despartamento de Ingeniería Electrónica y Telecomunicaciones\\
        Universidad de Antioquia\\
        Medellín\\
        Septiembre de 2020
            
    \end{center}
\end{titlepage}

\tableofcontents

\section{Sección introductoria}
Los desarrollos tecnológicos que se han obtenido a lo largo de la historia, han desempeñado un papel muy importante en nuestras vidas, pues nos han facilitado en gran medida el desarrollo de muchas actividades, por esto nos centraremos en uno de los avances tecnológicos mas importantes del siglo XX, el computador. 
A continuación procederemos con el estudio de uno de los componentes que integran el hardware, la "memoria". 

\section{Preguntas} \label{contenido}

\subsection{¿Qué es la memoria?} \label{contenido}
La memoria es un circuito electrónico que está compuesto por un conjunto de chips, ésta se encuentra presente en el hardware de nuestros dispositivos tecnológicos (como: PC, celulares, electrodomésticos, etc.). En ésta se guarda la información que van a ejecutar dichos dispositivos (por ejemplo, el sistema operativo), y funciona mediante señales eléctricas que comúnmente se suele denominar binario (0 cuando exista ausencia de voltaje y 1 para indicar que hay voltaje), la capacidad de la memoria la podemos medir en byte (que son la unión de 8 bits). La memoria (en este caso la de un computador), guarda de forma temporal la información que se obtiene de ejecutar las instrucciones que ingresa el usuario mediante los periféricos de entrada de la máquina (estas instrucciones se traducen a código binario, es decir, cadenas de ceros y unos que es lo que entiende el procesador). La memoria está en constante comunicación con la CPU quien recibe las instrucciones y datos de entrada que previamente fueron guardados en la memoria (todas la modificaciones o manipulación de la información no se efectúan directamente sobre la información contenida en el disco duro, si no en una copia que se encuentra en ejecución en la memoria). y a diferencia del disco duro,la memoria tiene mayor velocidad para procesar la información. 

\subsection{Tipos de memoria y descripcion de cada una de ellas}

\subsection{¿Como se gestiona la memoria en el computador?}

A continuación se presenta el logo de C++ Figura (\ref{fig:cpplogo})

\begin{figure}[h]
\includegraphics[width=4cm]{cpplogo.png}
\centering
\caption{Logo de C++}
\label{fig:cpplogo}
\end{figure}

En la sección de teoremas (\ref{contenido})

\section{Conclusión} \label{conclulsion}

\bibliographystyle{IEEEtran}
\bibliography{references}

\end{document}
